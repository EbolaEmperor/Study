%!TEX program = xelatex
% 完整编译: xelatex -> biber/bibtex -> xelatex -> xelatex
\documentclass[lang=cn,11pt,a4paper]{elegantpaper}

\title{Infiniband调研报告}
\author{W Huang}
\date{\zhtoday}


% 本文档命令
\usepackage{array}
\usepackage{float}
\usepackage{multirow}
\newcommand{\ccr}[1]{\makecell{{\color{#1}\rule{1cm}{1cm}}}}

\begin{document}

\maketitle

\section{需求描述}

我们当前并行计算的主要问题有两个:

\begin{enumerate}
    \item 单机计算内存带宽不够,导致数据阻塞,只有一两个核能充分工作。要直接解决这个问题,只能更换主板。
    \item 多机通讯采用了以太网,延迟太高,导致mpi函数很慢。
\end{enumerate}

本文关注第二个问题。我们现在的以太网设备速率是10Gbps,对当前项目来说已经足够,当然不排除未来需要更高速率的可能性。目前以太网不能满足延迟要求,我们考虑将网卡、网线、交换机全套设备更换成infiniBand。

我们采用osu benchmark进行测试,我们的以太网的通讯延迟大约是infiniBand的32倍。其中infiniBand的测试在学校超算平台进行。

\section{采购方案}

当前市场上infiniBand的品牌只有Mellanox一家,主流配置有100Gbps/200Gbps/400Gbps三种。考虑到我们的计算并不是运行在GPU上的数据密集型计算,采用100Gbps组网方案即可。采购单如下(南京星涌网络科技有限公司报价)

\begin{table}[H]
    \begin{tabular}{|c|c|c|c|}
    \hline
    \textbf{产品类型}       & \textbf{型号}     & \textbf{关键参数}                      & \textbf{单价(估)} \\ \hline
    交换机                 & MSB7800-ES2F    & 36口,单口100Gbps,吞吐量7Tbps,延迟90ns,136W & 89000          \\ \hline
    \multirow{3}{*}{网卡} & MCX653105A-ECAT & 6代产品,双口连接,100Gbps,延迟600ns          & 5300           \\ \cline{2-4} 
                        & MCX653106A-ECAT & 6代产品,单口连接,100Gbps,延迟600ns          & 4300           \\ \cline{2-4} 
                        & MCX555A-ECAT    & 5代产品,单口连接,100Gbps,延迟600ns          & 3900           \\ \hline
    网线                  & MFA1A00-E005    & 100Gbps,5米                         & 3800           \\ \hline
    \end{tabular}
\end{table}

选择最便宜的网卡,总估价为$88000+16\times 3900+16\times3800=211200$元。

双口的作用仅仅是防止意外损坏,不能让速度翻倍。另外Mellanox的产品从5代到6代只是把上限从100Gbps提高到了200Gbps,我们100Gbps的配置没必要买6代产品。

采购可以联系星涌网络销售人员(康亮,微信号luyeesk),或者咨询其它国内代理商。

\appendix
%\appendixpage
\addappheadtotoc

\end{document}
