\documentclass[lang=cn,10pt,bibend=bibtex]{elegantbook}

\title{空间自适应网格研究报告}

\author{Wenchong Huang}
\date{Oct. 10, 2023}

\setcounter{tocdepth}{3}

\logo{logo-blue.jpg}
\cover{cover.jpeg}
\usepackage{multirow}
\usepackage{xpatch}
\makeatletter
\xpatchcmd{\chapter}
  {\if@openright\cleardoublepage\else\clearpage\fi}{\par\relax}
  {}{}
\makeatother

% 本文档命令
\usepackage{array, float}
\newcommand{\ccr}[1]{\makecell{{\color{#1}\rule{1cm}{1cm}}}}

% 修改标题页的橙色带
% \definecolor{customcolor}{RGB}{32,178,170}
% \colorlet{coverlinecolor}{customcolor}

\begin{document}

\maketitle
\frontmatter

\tableofcontents

\mainmatter

\chapter{初步想法}

我们设想的网格自适应方式是:对于一个控制体,如果它自身或与它相邻的控制体被切割,那么把它等分成$2^D$份。重复这个过程,设定一个最细的网格宽度。

(Q1)网格如何编号?一种可能的方式如下图所示。这些格点用树形结构存储,然后按图示下标映射到一个列向量里。

(Q2)如何生成粗细网格交界处控制体的Stencil?一种可能的方案是基于PLG选点,再最小二乘拟合。另一种是填充Ghost Cells,然后用标准格式,但是如何填充Ghost?

(Q3)在边界附近的控制体如何处理?可以沿用切割边界控制体的方案。或者另一种选择是非贴体网格:因为自适应网格在边界附近已经足够细,可以考虑丢弃被边界切割的控制体。

(Q4)采用何种多重网格?可以沿用几何多重网格+块松弛的方案,但考虑到大部分控制体都集中在边界附近,它们的Stencil几乎都是不规则的,块松弛的代价可能太大。我们可能会考虑代数多重网格。

\end{document}