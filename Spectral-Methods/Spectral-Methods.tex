\documentclass[lang=cn,10pt,bibend=bibtex]{elegantbook}

\title{10月份科研报告}

\author{Wenchong Huang}
\date{Oct. 10, 2023}

\setcounter{tocdepth}{3}

\logo{logo-blue.jpg}
\cover{cover.jpeg}
\usepackage{multirow}
\usepackage{xpatch}
\makeatletter
\xpatchcmd{\chapter}
  {\if@openright\cleardoublepage\else\clearpage\fi}{\par\relax}
  {}{}
\makeatother

% 本文档命令
\usepackage{array, float}
\newcommand{\ccr}[1]{\makecell{{\color{#1}\rule{1cm}{1cm}}}}

% 修改标题页的橙色带
% \definecolor{customcolor}{RGB}{32,178,170}
% \colorlet{coverlinecolor}{customcolor}

\begin{document}

\maketitle
\frontmatter

\tableofcontents

\mainmatter

\chapter{分离方法}

\section{分离方法介绍}

假设我们有初值问题:
\begin{equation}
    \frac{d\phi}{dt}=A(\phi)+B(\phi),\quad \phi(0)=\phi_0.
\end{equation}

我们有求解器$\mathcal{N}_A(\phi_0,T)$,它输出初值问题
\begin{equation}
    \frac{d\phi}{dt}=A(\phi),\quad \phi(0)=\phi_0.
\end{equation}

在$T$时刻的解。同时也有求解器$\mathcal{N}_B(\phi_0,T)$,它输出初值问题
\begin{equation}
    \frac{d\phi}{dt}=B(\phi),\quad \phi(0)=\phi_0.
\end{equation}

在$T$时刻的解。我们可以借助这两个求解器构造一个分离格式,例如:
\begin{align*}
    \phi^\star &= \mathcal{N}_A(\phi_n,k),\\
    \phi_{n+1} &= \mathcal{N}_B(\phi^\star,k).
\end{align*}

这个格式有两个性质:
\begin{enumerate}
    \item 只要求解器$\mathcal{N}_A$与$\mathcal{N}_B$收敛,那么分离格式也收敛;
    \item 不论求解器$\mathcal{N}_A$与$\mathcal{N}_B$的精度有多高,在一些特定问题中,该分离格式也只有一阶精度。例如当$A,B$是不交换的线性算子时,可以证明:即使求解器都是精确的,其单步误差也将达到$O(k^2)$。
\end{enumerate}

\section{二阶时间离散:Strang Splitting}

我们沿用(1.1)(1.2)(1.3)式的符号。Strang Splitting格式\cite{Strang1968}如下:
\begin{align*}
    \phi^\star &= \mathcal{N}_A(\phi_n,k/2),\\
    \phi^{\star\star} &= \mathcal{N}_B(\phi^\star,k),\\
    \phi_{n+1} &= \mathcal{N}_A(\phi^{\star\star},k/2).
\end{align*}

为方便后文的表述,我们记由Strang Splitting格式算一步得到的解为:
\begin{equation}
    \phi_{n+1} = S(\phi_n,k).
\end{equation}

\section{四阶时间离散:Forest-Ruth splitting}

我们沿用(1.4)式的符号。Forest-Ruth Splitting格式\cite{Forest1990}如下:
\begin{align*}
    \phi^\star &= S(\phi_n,\omega_1k),\\
    \phi^{\star\star} &= S(\phi^\star,\omega_2k),\\
    \phi_{n+1} &= S(\phi^{\star\star},\omega_1k).
\end{align*}

其中
\begin{equation*}
    \omega_1=\frac{1}{2-2^{1/3}},\quad \omega_2=-\frac{2^{1/3}}{2-2^{1/3}}
\end{equation*}

众所周知,扩散方程的逆过程是不稳定的,而这个格式中居然出现了负时间步,它将会是万恶之源。

\section{四阶紧致时间离散:Chin splitting}

我们沿用(1.1)(1.2)(1.3)式的符号。同时我们设求解器$\mathcal{N}_C(\phi_0,\tau,T)$输出初值问题
\begin{equation}
    \frac{d\phi}{dt}=A(\phi)+\frac{\tau^2}{48}(2ABA-AAB-BAA)(\phi),\quad \phi(0)=\phi_0.
\end{equation}

在$T$时刻的解。Chin Splitting格式\cite{Chin1997}如下:
\begin{align*}
    \phi^{(1)} &= \mathcal{N}_A(\phi_n,k/6),\\
    \phi^{(2)} &= \mathcal{N}_B(\phi^{(1)},k/2),\\
    \phi^{(2)} &= \mathcal{N}_C(\phi^{(2)},k,2k/3),\\
    \phi^{(4)} &= \mathcal{N}_B(\phi^{(3)},k/2),\\
    \phi_{n+1} &= \mathcal{N}_A(\phi^{(4)},k/6).
\end{align*}

当$2ABA-AAB-BAA=0$时,这是一个相当好的格式。

\vspace{5em}

\chapter{谱方法}

\section{热方程的谱方法}

\section{对流扩散方程的有限体-谱分离方法}

\chapter{SAV方法}

\section{Cahn-Hilliard方程的SAV方法}

\section{GePUP-SAV-SDIRK}

\printbibliography[heading=bibintoc,title=\ebibname]

\appendix

\chapter{二维FFT}

\section{公式推导}

\begin{lemma}
    二维DFT可归结为如下问题:已知二元多项式
    \begin{equation*}
        f(x,y)=\sum_{n=0}^{N-1}\sum_{m=0}^{N-1} a_{n,m}x^ny^m,
    \end{equation*}

    求$f(\omega_N^j,\omega_N^k),(j,k=0,...,N-1)$的值。
\end{lemma}

引理很容易验证,关键是如何快速求解这个问题。为此,我们对多项式做奇偶项划分,即令
\begin{align*}
    p_1(x,y)&=\sum_{n=0}^{N/2-1}\sum_{m=0}^{N/2-1} a_{2n,2m}x^ny^m,\\
    p_2(x,y)&=\sum_{n=0}^{N/2-1}\sum_{m=0}^{N/2-1} a_{2n,2m+1}x^ny^m,\\
    p_3(x,y)&=\sum_{n=0}^{N/2-1}\sum_{m=0}^{N/2-1} a_{2n+1,2m}x^ny^m,\\
    p_4(x,y)&=\sum_{n=0}^{N/2-1}\sum_{m=0}^{N/2-1} a_{2n+1,2m+1}x^ny^m.
\end{align*}

于是我们有
\begin{equation*}
    f(x,y)=p_1(x^2,y^2)+yp_2(x^2,y^2)+xp_3(x^2,y^2)+xyp_4(x^2,y^2).
\end{equation*}

现任取整数$j,k\in[0,N/2)$,注意到
\begin{align*}
    f(\omega_N^j,\omega_N^k)&=p_1(\omega_{N/2}^j,\omega_{N/2}^k)+\omega_N^kp_2(\omega_{N/2}^j,\omega_{N/2}^k)+\omega_N^jp_3(\omega_{N/2}^j,\omega_{N/2}^k)+\omega_N^j\omega_N^kp_4(\omega_{N/2}^j,\omega_{N/2}^k)\\
    f(\omega_N^j,\omega_N^{N/2+k})&=p_1(\omega_{N/2}^j,\omega_{N/2}^k)-\omega_N^kp_2(\omega_{N/2}^j,\omega_{N/2}^k)+\omega_N^jp_3(\omega_{N/2}^j,\omega_{N/2}^k)-\omega_N^j\omega_N^kp_4(\omega_{N/2}^j,\omega_{N/2}^k)\\
    f(\omega_N^{N/2+j},\omega_N^k)&=p_1(\omega_{N/2}^j,\omega_{N/2}^k)+\omega_N^kp_2(\omega_{N/2}^j,\omega_{N/2}^k)-\omega_N^jp_3(\omega_{N/2}^j,\omega_{N/2}^k)-\omega_N^j\omega_N^kp_4(\omega_{N/2}^j,\omega_{N/2}^k)\\
    f(\omega_N^{N/2+j},\omega_N^{N/2+k})&=p_1(\omega_{N/2}^j,\omega_{N/2}^k)-\omega_N^kp_2(\omega_{N/2}^j,\omega_{N/2}^k)-\omega_N^jp_3(\omega_{N/2}^j,\omega_{N/2}^k)+\omega_N^j\omega_N^kp_4(\omega_{N/2}^j,\omega_{N/2}^k)\\
\end{align*}

因此,只要我们先算出
\begin{equation}
    p_i(\omega_{N/2}^j,\omega_{N/2}^k),(j,k=0,...,N/2-1),\quad i=1,2,3,4.
\end{equation}

我们就可以再用$O(N)$的时间完成问题求解。不难看出(A.1)式的形式与原问题完全一致,只是$N$换成了$N/2$。我们设求解原问题的时间复杂度为$T(N)$,有:
\begin{equation*}
    T(N)=4T\left(\frac{N}{2}\right)+\Theta(N)=\Theta(N^2\log N).
\end{equation*}

因此我们可以借助分治实现高效求解。但如果使用递归式写法,我们需要在每一层存储奇偶项划分后$p_1,p_2,p_3,p_4$的各项系数,从而空间复杂度也达到$\Theta(N^2\log N)$。我们可以用蝴蝶变换来避免递归,即自底向上模拟分治过程,空间复杂度可以下降到$\Theta(N^2)$。详见代码\verb|fft2D.cpp|。

\section{FFTW库的使用}

FFTW(FFT in the West)库曾经是世界上最快的FFT库,不过现在Intel用mkl优化后的FFTW要更快。以2维FFT为例,在输入输出均为复数数组的情况下,速度大约是我手写FFT的1.2倍。

注意到一个问题:FFT正变换时,输入数组为实数、输出数组为复数;FFT逆变换时,输入数组为复数、输出数组为实数。FFTW能利用这个性质,进一步优化FFT。下面是一个使用FFTW的例程。

\begin{lstlisting}[language=c++]
    #include <bits/stdc++.h>
    #include <fftw3.h>
    using namespace std;
    
    const int N = 1024;
    
    int main(){
        auto out = (fftw_complex*) fftw_malloc(sizeof(fftw_complex) * N*(N/2+1));
        auto in = (double*) fftw_malloc(sizeof(double) * N*N);
        auto p = fftw_plan_dft_r2c_2d(N, N, in, out, FFTW_MEASURE);
        auto pinv = fftw_plan_dft_c2r_2d(N, N, out, in, FFTW_MEASURE);
    
        for(int i = 0; i < N*N; i++)
            in[i] = (double)rand()/RAND_MAX;
        
        // Run FFT and save to out[]
        fftw_execute(p);
    
        // Run IFFT and save to in[]
        fftw_execute(pinv);
    
        return 0;
    }
\end{lstlisting}

实际测试表明,针对输入输出数组的类型优化后的FFTW,其运行速度是我手写FFT的2.4倍。

\newpage

\chapter{本月的心路历程}

本月搞懂了三维不规则有限体程序的脉络,并着手处理了一些程序细节。但是两位学长几乎没有给我安排代码任务,导致我在做完手头工作后有了空闲时间,于是开始读GePUP-SAV的文章。阅读文章的过程中,我看不懂SAV,然后就去找了SAV的原始论文来看。看完之后感觉很顺畅,于是想把文章中的数值测试简单复现一下。然后就开始了“递归式”的学习。

我发现文章中是用谱方法离散的,然后去看了一下谱方法。我发现谱方法要用到FFT,但我只学过1维FFT,网上也找不到2维FFT的资料,索性自己按1维的思路推了一遍。然后我用谱方法写了个二维规则周期区域热方程的求解器。有了这个求解器之后我又想把它用在对流扩散方程里,但非线性对流项在做Fourier变换之后反而比原来更复杂了。于是我就想用有限体处理对流项、用谱方法处理扩散项。为了把他们耦合起来,我又去看了些分离方法的文章。

看了一些文章,发现分离方法似乎现在已经没什么人研究了,于是回头把SAV原始文章里的数值测试复现了一下,接着回头读GePUP文章的SAV部分,然后这个月就这么过去了。

\end{document}