%!TEX program = xelatex
% 完整编译: xelatex -> biber/bibtex -> xelatex -> xelatex
\documentclass[lang=cn,11pt,a4paper]{elegantpaper}

\title{有限元第四次编程作业}
\author{W Huang}
\date{\zhtoday}


% 本文档命令
\usepackage{array}
\usepackage{float}
\usepackage{multirow}
\usepackage{amsmath}
\usepackage{amssymb}
\newcommand{\ccr}[1]{\makecell{{\color{#1}\rule{1cm}{1cm}}}}

\begin{document}

\maketitle

\section{Stokes方程}

\subsection{求解设置}

求解无滑移边界条件的不可压 Stokes 方程:
\begin{equation}
    \left\{
        \begin{array}{ll}
            -\Delta \mathbf{u}+\nabla p = \mathbf{f}& ,\text{in}\;\Omega,\\
            \nabla\cdot \mathbf{u}& ,\text{in}\;\Omega,\\
            \mathbf{u} = 0& ,\text{on}\;\partial \Omega.
        \end{array}
    \right.
\end{equation}
可以导出弱形式:
\begin{equation}
    (\nabla \mathbf{u},\nabla \mathbf{v}) 
    + (\nabla\cdot \mathbf{v},\nabla p)
    + (\nabla\cdot \mathbf{u},\nabla q)
    = (\mathbf{f},\mathbf{v}),\quad \forall \mathbf{v}\in H_0^1(\Omega),q\in L_0^2(\Omega).
\end{equation}
这里
\begin{equation}
    L_0^2(\Omega)=\left\{q\in L^2(\Omega):\int_\Omega q\;dx=0\right\}.
\end{equation}
现在我们在有限元空间 $\mathcal{P}_2^d\times \mathcal{P}_1$ 中取逼近。取精确解
\begin{equation}
    \mathbf{u}(\mathbf{x}) = \pi(\sin^2(\pi x_1)\sin(2\pi x_2),-\sin(2\pi x_1)\sin^2(\pi x_2)),
\end{equation}
\begin{equation}
    p(\mathbf{x}) = \cos(\pi x_1)\sin(\pi x_2),
\end{equation}
并导出右端项,进行测试。网格如图1所示。离散系统的稀疏模式 
(sparsity pattern) 具有明显的块状结构,
如图2所示(我们对自由度进行了重编号:先用 Cuthill McKee 
算法对所有自由度进行重排,
然后再保序地按 $u_1,u_2,p$ 的顺序重排)。

\begin{figure}[H]
    \centering
    \begin{minipage}[t]{0.4\textwidth}
        \centering
        \includegraphics[width=\linewidth]{fig/mesh-2D.png}
        \caption{\small $h=\frac{1}{8}$ 时的计算网格}
    \end{minipage}
    \hfill
    \begin{minipage}[t]{0.4\textwidth}
        \centering
        \includegraphics[width=\linewidth]{fig/sparsity-pattern-5.png}
        \caption{\small $h=\frac{1}{32}$ 时离散系统的稀疏模式}
    \end{minipage}
\end{figure}

按照题目要求,我们使用 MINRES 方法求解,\verb|deal.ii| 中为我们提供了
\verb|SolverMinRes|,可直接调用。现在考虑题目所述的预优算子:
\begin{equation}
    B=\begin{pmatrix}
        B_\Delta & &\\
        & B_\Delta &\\
        & & (\text{diag}\; M_Q)^{-1}
    \end{pmatrix}.
\end{equation}
其中 $B_\Delta$ 是使用V-多重网格进行一轮松弛。考虑到 $\mathcal{P}_2$ 元的
索引非常复杂,粗细网格插值极其困难,因此我们决定使用代数多重网格 (AMG)。
\verb|Trilinos| 为我们提供了相关的库,\verb|deal.ii| 将其引入并进行了
用户友好的封装,我们直接调用 \verb|TrilinosWrappers::PreconditionAMG| 即可。

最后,我们还需要计算误差。我们采用 $n=3$ 的高斯求积公式
(二维情形下,每个三角形中有 9 个积分节点,具有 5 阶代数精度)
来近似计算误差的 $L^2$ 范数。

\subsection{编译说明}

请安装依赖库:
\begin{itemize}
    \item \verb|openMPI|(也可以用 \verb|MPICH| 等其它 MPI 软件包代替);
    \item \verb|Trilinos| (\verb|deal.ii| 的 README 中提及了安装方式);
    \item \verb|deal.ii| (请确保 \verb|DEAL_II_WITH_TRILINOS| 开关是开启状态)。
\end{itemize}

在确保依赖库正确安装后,请输入以下命令编译。
\begin{lstlisting}
cd src
mkdir build
cd build
cmake ..
make release
make
\end{lstlisting}
等待编译完成后,用以下命令执行测试:
\begin{lstlisting}
./stokes
\end{lstlisting}
上述测试将从 $h=\frac{1}{4}$ 开始,逐次加密,
一直运行到 $h=\frac{1}{1024}$。

\subsection{测试结果}

\begin{figure}[H]
    \centering
    \begin{minipage}[t]{0.32\textwidth}
        \centering
        \includegraphics[width=\linewidth]{fig/velocity.png}
        \caption{\small $h=\frac{1}{256}$,速度场}
    \end{minipage}
    \hfill
    \begin{minipage}[t]{0.32\textwidth}
        \centering
        \includegraphics[width=\linewidth]{fig/velocity_mag.png}
        \caption{\small $h=\frac{1}{256}$,速度的大小 $|\mathbf{u}|$}
    \end{minipage}
    \hfill
    \begin{minipage}[t]{0.32\textwidth}
        \centering
        \includegraphics[width=\linewidth]{fig/pressure.png}
        \caption{\small $h=\frac{1}{256}$,压强 $p$}
    \end{minipage}
\end{figure}

我们将求得的数值解用 \verb|VisIt| 绘制,如图 $3-5$ 所示,
其图像与我们构造的解析解一致。

用朴素 MINRES 方法与预优 MINRES 方法对比,见表1。
表中的 “装配耗时” 是指从稀疏模式生成完毕开始,到刚度矩阵计算完成耗费的时间;
“求解耗时”是指 AMG 初始化和 MINRES 迭代的耗时之和。
可以看到朴素 MINRES 方法的迭代次数随着网格加密而指数增长,
求解时间也逐渐变得令人难以接受;
而预优 MINRES 方法的迭代次数涨幅非常小,
求解时间明显快于朴素 MINRES 方法,
甚至对 $h=\frac{1}{1024}$ 的网格也能轻松求解。

此外,我们还输出了预优 MINRES 方法求解误差的 $L^2$ 范数,
见表2。可以看到 $\mathbf{u}$ 保持了 $\mathcal{P}_2$ 元的
$L^2$ 范数三阶收敛性质;$p$ 保持了 $\mathcal{P}_1$ 元的
$L^2$ 范数二阶收敛性质。

% Please add the following required packages to your document preamble:
% \usepackage{multirow}
\begin{table}[H]
    \centering
    \begin{tabular}{c|ccccc}
    \hline
                                        & $h$               & $\frac{1}{128}$  & $\frac{1}{256}$ & $\frac{1}{512}$  & $\frac{1}{1024}$ \\ \hline
    \multirow{3}{*}{预优 MINRES 方法}    & 装配耗时 (s)     &   0.134   & 0.613            &   2.38   &  9.44              \\
                                        & 求解耗时 (s)       &   1.32   & 6.42            &   32.0   &  140              \\
                                        & MINRES 迭代次数        &   80   & 85             &   100   &  105             \\ \hline
    \multirow{3}{*}{朴素 MINRES 方法} & 装配耗时 (s)        &  0.168    & 0.601         &  2.37    &  9.59             \\
                                    & 求解耗时 (s)         &   9.61   &   93.5     &   791   &    >1h, killed   \\
                                    & MINRES 迭代次数          &   2898   &  5449     &   10628   &                \\ \hline
    \end{tabular}
    \caption{\small 预优 MINRES 方法与朴素 MINRES 方法,$h=\frac{1}{128}$ 到 $\frac{1}{1024}$ 的 CPU 耗时。迭代终止条件均为残差的 $2$ 范数小于等于右端项 $2$ 范数的 $10^{-6}$ 倍,即 $||\text{res}||_2\leq 10^{-6}||\text{rhs}||_2$。}
\end{table}

\begin{table}[H]
    \centering
    \begin{tabular}{cccccc}
    \hline
         $h$               & $\frac{1}{256}$ & 收敛阶  & $\frac{1}{512}$ & 收敛阶  & $\frac{1}{1024}$ \\ \hline
         $||u_1^h-u_1||_{L^2}$ & 2.03e-07        & 2.99 & 2.56e-08        & 2.91 & 3.40e-09         \\
         $||u_2^h-u_2||_{L^2}$ & 2.03e-07        & 2.98 & 2.57e-08        & 2.89 & 3.46e-09         \\
         $||p^h-p||_{L^2}$     & 6.29e-06        & 1.99 & 1.58e-06        & 1.97 & 4.02e-07         \\ \hline
    \end{tabular}
    \caption{\small 预优 MINRES 方法,$h=\frac{1}{256}$ 到 $\frac{1}{1024}$ 的 $L^2$ 误差}
\end{table}

\appendix
%\appendixpage
\addappheadtotoc

\end{document}
