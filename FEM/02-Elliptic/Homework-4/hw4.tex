%!TEX program = xelatex
% 完整编译: xelatex -> biber/bibtex -> xelatex -> xelatex
\documentclass[lang=cn,11pt,a4paper]{elegantpaper}

\title{有限元第二次编程作业}
\author{W Huang}
\date{\zhtoday}


% 本文档命令
\usepackage{array}
\usepackage{float}
\usepackage{multirow}
\newcommand{\ccr}[1]{\makecell{{\color{#1}\rule{1cm}{1cm}}}}

\begin{document}

\maketitle

\section{编程第一题}

\subsection{求解设置}

求解PDE
\begin{equation}
    \left\{
        \begin{array}{l}
            -\Delta u = f,\quad \text{in}\;(0,1),\\
            u(0) = u(1) = 0.
        \end{array}
    \right.
\end{equation}

取一个 $L^2$ 可积但在 $0$ 处不连续的右端项:
\begin{equation}
    f(x)=\ln x.
\end{equation}

导出精确解:
\begin{equation}
    u(x)=\frac{1}{2}x^2\ln x - \frac{3}{4}x^2 + \frac{3}{4}x.
\end{equation}

使用非均匀网格$x_i=(i/N)^2$,取$\mathcal{P}_1$元。使用预优共轭梯度法 (Preconditioned CG) 求解,用超松弛迭代 (SSOR) 作为预优因子,超松弛系数取 $1.99999999$。

右端项的数值积分由一阶高斯求积公式计算。

\subsection{数值结果}

\begin{table}[H]
    \centering
    \begin{tabular}{|c|c|c|c|c|c|c|c|}
    \hline
    单元数量                    & $2^{14}$ & 阶数 & $2^{15}$ & 阶数 & $2^{16}$ & 阶数 & $2^{17}$ \\ \hline
    $||u-u_h||_{L_2}$      & 5.10e-10     & 2.05 & 1.23e-10     & 2.24 & 2.61e-11     & - & 5.47e-11     \\ \hline
    $||u-u_h||_{L_\infty}$ & 8.42e-10     & 2.04 & 2.05e-10     & 2.04 & 4.98e-11     & - & 9.28e-11     \\ \hline
    $||u-u_h||_{H_1}$      & 1.76e-05     & 1.00 & 8.81e-06     & 1.00 & 4.40e-06     & 1.00 & 2.20e-06     \\ \hline
    CG 迭代次数            & 18 & & 19 & & 20 & & 21\\    
\hline
    装配耗时 (s)           & 0.024         &      & 0.050          &      & 0.067           &      & 0.14     \\ \hline
    求解耗时 (s)           & 0.0079         &      & 0.014          &      & 0.031           &      & 0.052     \\ \hline
    \end{tabular}
    \caption{\small 预优共轭梯度法,预优因子:SSOR,超松弛系数:$1.99999999$。}
\end{table}

收敛阶令人满意。由于网格尺寸太细,在机器精度的限制下,$L_2$ 和 $L_\infty$ 范数已经无法继续下降。另外可以看到,SSOR 作为预优因子效果非常好,随着网格加密,CG 迭代次数基本不会增加。换言之,当超松弛系数趋近于 $2$ 时,在 SSOR 的作用下,迭代矩阵的条件数与网格尺寸几乎无关。

刚度矩阵条件数(二范数下)的数值结果如下,数值结果支持 $\kappa(A)\sim O(N^3)$:

\begin{table}[H]
    \centering
    \begin{tabular}{|c|c|c|c|c|c|c|c|}
    \hline
    单元数量                    & $256$ & 阶数 & $512$ & 阶数 & $1024$ \\ \hline
  $\kappa(A)$            & 1.93116e+06 & 2.99 & 1.53591e+07 & 3.00 & 1.22513e+08\\    \hline
    \end{tabular}
    \caption{\small 刚度矩阵的二范数条件数,即 $\kappa(A)=||A||_2\cdot ||A^{-1}||_2$。}
\end{table}

为了测试刚度矩阵的条件数对求解性能的影响,我们不使用预优因子再进行一次测试。与预优 CG 相比,朴素 CG 的求解性能大大降低,我们只好将网格规模减小以进行测试。

\begin{table}[H]
    \centering
    \begin{tabular}{|c|c|c|c|c|c|c|c|}
    \hline
    单元数量                    & $2^{12}$ & 增长率 & $2^{13}$ & 增长率 & $2^{14}$ \\ \hline
    CG 迭代次数            & 42996 & 2.86 & 122781 & 2.85 & 350164\\    
\hline
    装配耗时 (s)           & 0.005          &      & 0.01           &      & 0.02     \\ \hline
    求解耗时 (s)           & 0.49          &      & 2.35           &      & 12.7     \\ \hline
    \end{tabular}
    \caption{\small 朴素共轭梯度法。}
\end{table}

\appendix
%\appendixpage
\addappheadtotoc

\end{document}
